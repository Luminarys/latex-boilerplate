
\usepackage{float}
\usepackage{graphicx}
\usepackage{setspace}
% \usepackage{kpfonts}
\usepackage{textcomp}
\usepackage{fullpage}
\usepackage{url}

% -- font styles
% \renewcommand*\rmdefault{dayrom}
\usepackage[sc]{mathpazo} % Palatino (very readable)
\usepackage[T1]{fontenc}
% \usepackage{gfsdidot}
% \usepackage[scaled]{beraserif}
% \usepackage[bitstream-charter]{mathdesign}

\floatstyle{ruled}

% -- structural elements
\newfloat{program}{thp}{lop}
\floatname{program}{Program}

\newfloat{figure}{thp}{lop}
\floatname{figure}{Figure}

% -- syntax highlighting
\usepackage{listings}
  \usepackage{courier}
\usepackage{color}


% \onehalfspacing
\setstretch{1.08}
\parskip 9pt

% page margins
\setlength{\textheight}{23cm}
\setlength{\oddsidemargin}{0.25in}
\setlength{\textwidth}{6in}

% I don't know what this does
\def\printcitestart{\unskip $^\bgroup}
\def\printbetweencitations{,}
\def\printcitefinish{\egroup$}
\def\printcitenote#1{\hbox{\sevenrm\space (#1)}}

\newenvironment{aside} {
  \parindent 7.2pt
  \parskip 5pt
} { }


\lstset{
  upquote=true,
  breaklines=false,
  %postbreak=\raisebox{0ex}[0ex][0ex]{\ensuremath{\hookrightarrow}},
  breakatwhitespace=true,
  %numbers=left,
  language=Java,
  basicstyle=\footnotesize\ttfamily,
  %numberstyle=\tiny,
  %numbersep=5pt,
  tabsize=2,
  extendedchars=true,
  showtabs=false,
  showspaces=false,
  showstringspaces=false
}
\lstloadlanguages{
  Java
}


%% \hyphenation{Fire-Detection-System}
%% \hyphenation{Emergency-Communication-System}
